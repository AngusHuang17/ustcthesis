
\chapter{绪论}
\label{chap:introduction}

中国科学技术大学,中国科学技术大学,中国科学技术大学,中国科学技术大学,中国科学技术大学,中国科学技术大学
中国科学技术大学,中国科学技术大学,中国科学技术大学


中国科学技术大学

\section{系统要求}



\section{下载与安装}


\href{http://code.google.com/p/ustcthesis}{http://code.google.com/p/ustcthesis}




\section{问题反馈}

用户在使用中遇到问题或者需要增加某种功能,请到\href{http://bbs.ustc.edu.cn/cgi/go?cgi=bbsdoc&board=TeX}{瀚海星云bbs,Tex版}反映。


欢迎大家反馈自己的使用情况,使我们可以不断改进宏包。

\section{为人民服务}


我们的共产党和共产党所领导的八路军、新四军,是革命的队伍。我们这个队伍完全是为着解放人民的,是彻底地为人民的利益工作的。张思德同志就是我们这个队伍中的一个同志。

人总是要死的,但死的意义有不同。中国古时候有个文学家叫做司马迁的说过:人固有一死,或重于泰山,或轻于鸿毛。为人民利益而死,就比泰山还重;替法西斯卖力,替剥削人民和压迫人民的人去死,就比鸿毛还轻。张思德同志是为人民利益而死的,他的死是比泰山还要重的。因为我们是为人民服务的,所以,我们如果有缺点,就不怕别人批评指出。不管是什么人,谁向我们指出都行。只要你说得对,我们就改正。你说的办法对人民有好处,我们就照你的办。“精兵简政”这一条意见,就是党外人士李鼎铭⑶先生提出来的;他提得好,对人民有好处,我们就采用了。只要我们为人民的利益坚持好的,为人民的利益改正错的,我们这个队伍就一定会兴旺起来。

我们都是来自五湖四海,为了一个共同的革命目标,走到一起来了。我们还要和全国大多数人民走这一条路。我们今天已经领导着有九千一百万人口的根据地,但是还不够,还要更大些,才能取得全民族的解放。我们的同志在困难的时候,要看到成绩,要看到光明,要提高我们的勇气。中国人民正在受难,我们有责任解救他们,我们要努力奋斗。要奋斗就会有牺牲,死人的事是经常发生的。但是我们想到人民的利益,想到大多数人民的痛苦,我们为人民而死,就是死得其所。不过,我们应当尽量地减少那些不必要的牺牲。我们的干部要关心每一个战士,一切革命队伍的人都要互相关心,互相爱护,互相帮助。

今后我们的队伍里,不管死了谁,不管是炊事员,是战士,只要他是做过一些有益的工作的,我们都要给他送葬,开追悼会。这要成为一个制度。这个方法也要介绍到老百姓那里去。村上的人死了,开个追悼会。用这样的方法,寄托我们的哀思,使整个人民团结起来。